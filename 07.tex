\newpage
\section{関連研究}
Feltenら~\cite{felten2000timing}は,ウェブページのロード時間によって,ブラウザの閲覧履歴を採取する手法を提案した.ブラウザまたはDNSのキャッシュを照会するタイミングアタックを実施し,閲覧履歴の推定精度は90%以上であったことを示した.

Kohnoら~\cite{kohno2005remote}は,TCPパケットを用いた識別手法を提案した.TCPタイムスタンプを利用することで,端末の時刻のずれ\footnote{Clock Skew}を採取した.これはNATまたはファイアウォールの内側にある場合においても情報を採取される可能性を示している.

Mayer~\cite{mayer2009any}は,インターネットでの匿名性について言及した.特にBrowser Fingerprintingに関しては最も初期に論じている.JavaScriptのオブジェクトやプラグイン,MIMEタイプなどから得られる情報により,インターネットにおける匿名性が低減されることを明らかにした.

Eckersley~\cite{eckersley2010unique}は,Fingerprintingを実施するウェブサイトを構築し分析を行った.その結果,約30万サンプルの83.6\%がユニークであり,Flash やJavaを実行できる端末に限定した場合は94.2\%がユニークであることを示した.EckersleyがFingerprintingを発表して以来,これに関する技術が多く提案されている.

Moweryら~\cite{mowery2011fingerprinting}は,JavaScriptの様々な処理のパフォーマンスによるベンチマークを用いて,同一端末上での複数のブラウザファミリの特定と,OS,CPUの識別を行い,実験結果を示している.ブラウザファミリーは98.2\%,CPUアーキテクチャは45.3\%の精度で識別できることを示した.

Mayerら~\cite{mayer2012third}は,サードパーティによるインターネット広告に関するプライバシー問題の調査を実施した.サードパーティウェブトラッキングに関する技術としてSuper Cookieの仕組みについて解説し,それらの追跡から保護するシステムについて提案を行っている.Super Cookieに関しては後にHTST~\footnote{HTTP Strict Transport Security}やHPKP~\footnote{Public Key Pinning Extension for HTTP}を利用したものが実施されている.

Nikiforakisら~\cite{nikiforakis2013cookieless}は,Fingerprintingを利用しているサイトを調査し,Alexa TOP10000サイトの0.4\%が該当することを明らかにした.また,Panopticlickと大手Fingerprintingプロバイダ3社の手法を比較し解説している.さらに,JavaScriptオブジェクトに特別な操作を行い,Fingerprintingへの活用を示した.また,ブラウザ拡張機能を使用した場合にブラウザの特徴が偽装できているかの実験結果も示し,偽装ツールを用いた場合に逆に追跡されうるリスクがあることに言及している.

Mulazzaniら~\cite{mulazzani2013fast}は,JavaScriptエンジンの実装状況を利用しブラウザの識別を行った.test262と呼ばれるテストスイートを用いて,ブラウザファミリとバージョンを判別している.

Moweryら~\cite{mowery2012pixel}は,HTML5のCanvas APIやWebGLを用いて文字列や画像を描画し,描画結果の画素レベルの差からグラフィックボードなどの情報を抽出した.これはGPU,OS,ブラウザの組合せによって変化する.300個のサンプルを収集した際のエントロピーは,5.73ビットとある.

Bojinovら~\cite{bojinov2014mobile}は,2種類のFingerprintingを提案した.スピーカーフォンまたはマイクの周波数応答を利用するFingerprinting,端末の加速度センサーのエラーに依存するFingerprintingの2種類である.User Agent文字列と組み合わせることで,53%の精度で1900の端末を識別できることを示した.

Nmap~\cite{nmap}やp0f~\cite{p0f}に代表されるツールはTCP/IPにおいてパケットを解析することでOSを推定する.これは特にOS Fingerprintingと呼ばれている.また,JavaScriptのMathオブジェクトを利用してOSが推定できることも報告されている~\cite{tor_bugtrack}.

Nakiblyら~\cite{nakibly2015hardware}は,HTML5のAPIからアクセスできるハードウェアに関して調査した.さらに,WebGLを用いてブラウザ上で図形を描画しGPUのクロックレートとクロックスキューを採取する初めての試みを実施した.その結果,同一構成の異なる端末においても違いが発生し,GPUパフォーマンスを採取するにはソフトウェアの影響を取り除く必要性があると述べている.また,HTML5 APIからアクセスできるハードウエア情報をまとめ,GPU,カメラ,スピーカーやマイク,モーションセンサー,GPS,バッテリーを用いたFingerprintingについて紹介している.

後藤ら~\cite{後藤浩行2013web}は,CPUのコア数,GPUレンダリングの有無,メディアデバイスの有無や,タブレット端末の表示向きといった複数のハードウエア情報をFingerprintingで採取できることを示した.

桐生らは,演算結果の差によってCPUの拡張命令であるSSE2~\footnote{Streaming SIMD Extensions 2}への対応の有無が特定できることを示した~\cite{桐生直輝2013cpu}.また,Web Worker APIを用いてマルチスレッド処理を行い演算速度の差を測定することでCPUコア数が推定できることを示した~~\cite{桐生直輝2014web}.

Saitoらは,JavaScriptのベンチマークを詳細に分析し,CPUの機能のTurbo BoostとCPUの拡張命令のAES-NIの推定手法を提案した~\cite{saito2016estimating}.
また,CPUの識別手法も提案し,モデルナンバーを91.4\%の精度で識別できると示した~\cite{saito2017web}.

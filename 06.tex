\newpage
\section{考察}
\subsection{\hfp~の目的}
\hfp~は\fp~と同様の利用法の他に独自の利用例があると考えられる.
一つは特定のハードウェアを搭載した端末に対するマーケティングである.
正確に端末構成を把握することで端末を扱う上で生じうる需要の予測が容易になると考えられる.
例えば,Hardware FingerprintingをWeb広告事業者が実施することで,ユーザに対して特定のハードウェアに関する商品を勧めるターゲティング広告が想定される.

もう一つは,端末構成に脆弱性を抱えたデバイスが含まれているかを検出され攻撃に利用されることが挙げられる.
アクセスしてきた端末がVMかどうかを識別し,悪意のあるサイトに誘導するという手法もがHoら\cite{ho2014tick}によって提案されているが,同様に特定のデバイスを持つ端末のみを脆弱性を利用した攻撃を行うサイトに誘導されるリスクは否めない.
これらに利用される場合はユーザにとって好ましいとはいえないので,ユーザ側で実施できるような対策が必要となる.

さらに,\fp~は時間経過に応じて識別が困難になることが指摘されている\cite{磯侑斗2014web}.
ハードウェアはソフトウェアに比べてアップデートによる変更があるわけではないので,\hfp~による長期的な識別も利用法として想定される.

\subsection{\hfp~の対策}
ブラウザには処理の高速化や利便性の向上を目的として,低レイヤなハードウェア資源へのアクセスを可能とする実装が存在する.
本研究で提案した\hfp~でも\texttt{Typed Array}を利用することでC言語のようにアドレスを指定して値に直接アクセスできることを示した.
他にも,端末に接続されたUSBデバイスへのアクセスが可能となるWeb USB API\cite{webusb}やWeb端末に搭載されているメモリサイズを得られるDevice Memory API\cite{device_memory}などの実装が検討されており,低レイヤのハードウェア資源へのアクセスが可能になりつつある.また,過去にはバッテリー情報を取得するAPIが存在したが,Diazら\cite{diaz2015leaking}の報告により削除されたこともあった.

これらの低レイヤ資源のアクセスは,Webサーバが端末固有の情報を得ることを可能とする.
意図しないAPIの実行を防ぐためには,ユーザが前節で述べたようなリスクについて理解し,かつ低レイヤなハードウェア資源へのアクセスを実施する際にはオプトインでユーザへ許諾を取る必要性があると考えられる.

\subsection{提案手法ついて}
本節では,本研究における貢献と対策,今後の展開について述べる.

\subsubsection{本研究の貢献}
採取システムの構築は,端末構成を採取した後にシームレスに\hfp~を行うことを可能にした.
その結果,端末構成を推定し検証することを容易にした.
さらに,正確な端末構成情報が提供されるので,非標本誤差を低減することができる.

提案手法の多くは高い正答率で,端末構成を推定することができたが,推定にかかる時間は長い.
しかし,リスクベース認証やデジタル・フォレンジックのように高い識別精度が求められる場合には有用性が高いとも言える.
用途に応じて,利用する特徴点を変更するなどして柔軟な対応をすることで実用的なものになると考えられる.

\subsubsection{提案手法に対する対策}
提案手法の多くに共通するのが,時刻計測を行うことである.
時間計測を行うことができる関数として,JavaScriptの標準ビルドインオブジェクトであるDateとHigh Resolution Time API\cite{high_res_time}がある.
Dateはミリ秒単位の精度,High Resolution Time APIではマイクロ秒単位の精度で計測が可能となっている.
提案手法ではこれらの関数を用いているので,精度を低下させると正確な推定が行えない.対策を行う際には,Dateで計測可能な精度のミリ秒単位よりも落とす必要がある.
ただし,JavaScriptの標準ビルドインオブジェクトの書き換えはWebサイト上にあるほかのコンテンツへ影響を与える可能性が高い.
また,High Resolution Time APIを無効化することでアプリケーションのパフォーマンスや応答性に悪影響を及ぼすことがある.
Torに関しては,\texttt{dom.enable\_performance}によりHigh Resolution Time APIを無効にすることで高精度な時間測定を防止している.

一方で,既存の\fp~に対する対策手法は有効ではない可能性が高い.
FireglovesやSecret AgentなどのFingerprinting対策手法の大部分は,時刻計測以外のJavaScriptのプロパティやオブジェクトの出力を書き換えるのみなので効果的ではない.

\subsubsection{今後の展開}
採取システムに関して,端末構成の採取をSMBIOSのみに頼っている.
SMBIOSからはBIOSが管理する情報のみしか採取できない.
よって,GPUのようにPCIeで接続するデバイスに関しては,PCIバスに接続されているか否かはSMBIOSから採取可能であるが,接続先のGPUの情報を得ることができない.
PCIe接続されたデバイスに関しては,\texttt{lspci}を用いることでGPUの詳細な情報が採取できる.

また,本研究で対象とした端末はWindowsとmacOSのみである.
Linuxやモバイル,タブレットといった環境においても情報の取得が行えるようにプログラムを改変することで網羅的な調査が実現する.

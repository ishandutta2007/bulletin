\newpage
\section{Browser Fingerprinting}
\subsection{定義}
Browser Fingerprintingの説明に関わる以下の用語について定義する.
\begin{itemize}
 \item Fingerprint
 \item Fingerprinting
 \item 特徴点
\end{itemize}
RFC 6973~\cite{rfc6973}では,Fingerprintを以下の通り定義している.
\begin{quote}
A set of information elements that identifies a device or application instance.
\end{quote}
つまり,Fingerprintは端末やアプリケーションインスタンスを識別する情報要素の集合を指す.例えば,アプリケーションインスタンスがウェブブラウザの場合は,User-Agent文字列の情報(Mozilla/5.0 (Windows NT 10.0; Win64; x64) AppleWebKit/537.36 (KHTML, like Gecko) Chrome/63.0.3239.132 Safari/537.36)が該当する.また,Fingerprintingは,以下の通り定義している.
\begin{quote}
The process of an observer or attacker uniquely identifying (with a sufficiently high probability) a device or application instance based on multiple information elements communicated to the observer or attacker.
\end{quote}
つまり,FingerprintingはFingerprintによって端末やアプリケーションインスタンスを高い確度で識別する行為を指す.
また,特徴点はFingerprintの各要素の名称と本研究では定義する.例えば,User-Agent文字列,IPアドレスが該当する.

ここで,Browser Fingerprintingは,RFC 6973におけるアプリケーションインスタンスがブラウザである場合と定義する.ブラウザから採取できるFingerprintによる識別がBrowser Fingerprintingである.また,本研究において,採取する際にユーザによる操作が必要ではない特徴点を扱うこととする.例えば,Geolocation APIによって,利用者の端末の位置情報を取得しようとすると,ブラウザは利用者に許可を求める.このように,ユーザによる操作が必要となる情報はFingerprintとして扱わない.

\subsection{特徴点の分類}
\subsubsection{ハードウェア特徴点}
\subsubsection{ソフトウェア特徴点}
\subsubsection{ネットワーク特徴点}
\subsection{利用法}
\subsection{対策技術}

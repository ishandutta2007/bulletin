\newpage
\section{まとめ}
本研究では\fp~を拡張した\hfp~について定義し,その採取システムを構築した.
これにより,端末構成を推定する仕組みが整えられ,既存の\fp~での利用法以外の活用についても議論した.

また,新たなハードウェア特徴点の採取法を提案し,CPUの拡張機能やCPUファミリ,マイクロアーキテクチャ,モデルナンバー,バイトオーダー,メモリのパフォーマンス,GPUのパフォーマンスの推定を実施した.
一部の推定については採取システムの一部で取得したSMBIOSとの比較を行った.
システムの構築は先行研究で問題となった教師データの非統計誤差を低減させることに貢献する.

今後の課題としては大規模にデータを収集する環境を整えた上で大量の教師データを確保し,確度の高い端末構成が行える状態にする必要があると考えられる.また,端末構成の推定が及ぼす影響については議論を重ね続ける必要があるであろう.
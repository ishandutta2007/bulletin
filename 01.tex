\section{はじめに}
World Wide Webの発展により今日ではインターネット上での情報交換が爆発的に広まることとなった. World Wide Webを利用に必要なウェブブラウザは1993年にリリースされたNCSA Mosaic\footnote{\url{http://www.ncsa.illinois.edu/enabling/mosaic}}により一般社会に普及した.NCSA Mosaicはテキストと画像を混在した基礎的なウェブページの表示のみができた.1995年に動的なウェブページを実現する目的でNetscape Navigator 2.0\footnote{2008年にすべてのバージョンのサポートが終了}に簡易的なプログラミング言語としてJavaScriptが実装された.その後,ECMA~\footnote{情報通信システムの国際標準化団体~\url{http://www.ecma-international.org/}}により標準化が行われECMA-262として仕様が公開されている~\footnote{\url{https://www.ecma-international.org/publications/standards/Ecma-262.htm}}.

一方で,World Wide Webの発展に伴い商業的な利用も活発化し,その例として電子商取引や広告配信が挙げられる.インターネット上での広告配信はウェブサイトに訪問したユーザの趣味嗜好に適したものが選択されるターゲティング広告という手法が存在する.これは,状態を保存しないHTTPの仕様において,状態を保持するHTTP Cookieという仕組みを利用する.HTTP Cookieを利用することでユーザがどのウェブサイトに訪問したかを取得でき,趣味嗜好の分析に活用できる.ところが,ウェブサイトへの訪問履歴をユーザの許可なく取得する行動追跡を行うことはプライバシー保護の観点から問題視されていた.2009年にEUでは,プライバシー保護規則のE-Privacy Directiveの改正によって,ユーザの行動追跡を目的としてHTTP Cookieを利用する場合はユーザの許可が必要なオプトイン方式であることを定めた.さらに,2011年にW3CはHTTP Cookieによる行動追跡を希望しない旨をウェブサイトに通知するDNT~\footnote{Do Not Track~\url{https://www.w3.org/TR/tracking-dnt/}}が仕様化された.これに対して,HTTP Cookie以外のウェブブラウザの保存機構を用いた追跡技術であるSuper Cookieが利用され始め,ブラウザベンダは適宜対応している. 

このような背景の中で,2010年にEFF~\footnote{Electronic Frontier Foundation}はインターネット上でのユーザ行動追跡に関する新たな研究としてPanopticlick Projectを開始した.ユーザがウェブサイト訪問の際に使用したウェブブラウザから採取可能な情報を組み合わせることで識別を行うことができるという報告がなされた~\cite{eckersley2010unique}.この技術はBrowser Fingerprintingと呼ばれ,HTTPやJavaScriptによって情報の取得を行う.実験においてこの技術で採取されたデータの94.2\%がユニークとなった.この研究を機にBrowser Fingerprintingへの研究が盛んに行われるようになった.2018年現在ではE-Privacy Directiveの後継としてEPrivacy Regulationが策定中であり,この草稿ではBrowser Fingerprintingに関する記載もなされていることから一般社会からの関心もあることが伺える.

Browser Fingerprintingはウェブブラウザに依拠した情報を扱うことから,ユーザが使用するウェブブラウザを変更した際には識別を行うことはできない.これに関して,ウェブブラウザを横断して識別を行うCross Browser Fingerprintingといった研究もなされている.本研究ではさらに踏み込んで,ユーザが使用する端末に依拠した情報の採取を試みた.これにより,Browser Fingerprintingによる識別可能性が高まるだけではなく,ユーザの行動追跡以外の利用法が想定される.以降,新たに提案する端末構成情報の採取手法と本研究で作成した端末構成を推定するシステムについて詳細に述べていく.
